\phantomsection\section*{Preface}\label{sec:preface}
\addcontentsline{toc}{section}{Preface}

%\vspace*{-10pt}\rule{\textwidth}{1.6pt}\vspace*{-\baselineskip}\vspace*{2pt}
%\rule{\textwidth}{0.4pt}\\\par

This is the preface to the first edition of \textit{Permutations:~A Beginners Guide}. This book is intended to be a soft introduction to permutation cycles and their common notations, as many students struggle to fully grasp the rules of permutations. %practise is a verb and practice is a noun

The content of chapter one should be accessible to all that read this book with some formal maths knowledge (A Level Mathematics or equivalent would suffice).

The following chapters get heavier and begin to introduce content taught to undergraduates.

There is usually some freedom of choice with calling a result a \textit{claim} or a \textit{lemma} or a \textit{theorem}. For clarity, we shall define these terms in the following way.

\begin{table}[h]
    \centering
    \begin{tabular}{ll}\hline\\[-15pt]\hline\\[-12pt]
        Claim:      & A small result useful for proving a stronger result like a lemma or a theorem.\\
        Lemma:      & An interesting or useful result.\\
        Theorem:    & An important result.\\
        Corollary:  & A result that (typically) follows easily from a lemma or a theorem.\\[2pt]
        \hline\\[-15pt]\hline\\
    \end{tabular}
\end{table}\vspace*{-20pt}

Of course, the subjectivity of what constitutes `important' is one reason why the same result may be labelled inconsistently. However, this is all cosmetic -- what really matters is the truth of the statements.