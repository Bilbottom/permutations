\section{The Symmetric Group}\label{sec:theorems}%

Now that the main notation and conventions have been established, some of the main theorems and permutation structures can be introduced. The following chapters are more technical than the first.

\subsection{Groups}\label{subsec:groups}

Informally, as mentioned in the preliminaries, a group is a set of elements with some way of combining elements to give another element in the set. We now introduce the formal definition of a group.

\begin{definition}[(Group)]
    A group \(G\) is a set together with a binary operation, usually called \textit{multiplication}, such that the following hold:
    \vspace*{-10pt}\begin{enumerate}[nolistsep]
        \item[--] for all \(g_{1}, g_{2}, g_{3} \in G\) we have \((g_{1}g_{2})g_{3} = g_{1}(g_{2}g_{3})\);\hfill(associativity)
        \item[--] there is an element \(1_{G} \in G\) such that \(g1_{G} = 1_{G}g = g\) for all \(g \in G\);\hfill(identity element)
        \item[--] for all \(g \in G\) there exists \(g^{-1} \in G\) such that \(gg^{-1} = g^{-1}g = 1_{G}\).\hfill(inverses)
    \end{enumerate}\vspace*{-10pt}
    We write \(1\) instead of \(1_{G}\) if it is clear to do so. The trivial group \(\{1\}\) with order \(1\) is also denoted \(1\).
\end{definition}

\subsection{Notable Lemmas}

We will use the terminology \textit{product} of permutations to mean either a product or a composition of permutations, as it will be made clear which reading convention will be used. The following fact has been used implicitly without mention, but will be formally proved now.

\begin{lemma}
    A product of permutations is a permutation.
\end{lemma}

\begin{definition}[(Transposition)]
    A permutation cycle with exactly two elements, say \((a\quad b)\), is called a \textit{transposition}.
\end{definition}

\begin{lemma}
    Every permutation can be decomposed into a product of transpositions.
\end{lemma}

\begin{definition}[(Permutation Parity)]
    A permutation is an \textit{even} permutation if it can be decomposed into an even number of transposition. A permutation is an \textit{odd} permutation if it can be decomposed into an odd number of transpositions.
\end{definition}

The \textit{parity} of a permutation is whether the permutation is even or odd. The proof of the following lemma shows that it must be one or the other, and cannot be both.

\begin{lemma}
    Every permutation is either even or odd, and is never both.
\end{lemma}

An alternate description of parity is to define the \textit{sign} of a permutation.

\begin{definition}[(Permutation Sign)]
    Suppose that \(\sigma\) is a permutation that can be decomposed into \(k\) transpositions. Then the \textit{sign} of \(\sigma\), written \(\sgn(\sigma)\), is equal to \((-1)^{k}\).
\end{definition}

We see that \(\sgn(\sigma) = 1\) is \(\sigma\) is even, and \(\sgn(\sigma) = -1\) if \(\sigma\) is odd.

\begin{lemma}
    The sign of a permutation is either \(-1\) or \(1\), and is never both.
\end{lemma}

\begin{lemma}
    The sign of a permutation is \(-1\) if and only if the permutation is odd. The sign of a permutation is \(1\) if and only if the permutation is even.
\end{lemma}

In Exercise~\ref{ex:ch1-powers} we encountered the \textit{identity permutation}, which maps every element to itself.

\begin{lemma}
    The identity permutation is an even permutation. Every transposition is odd.
\end{lemma}
